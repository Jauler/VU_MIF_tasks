\documentclass[12pt, a4paper, lithuanian, final]{article}

\usepackage{hyperref}
\usepackage{graphicx}
\usepackage{float}
\usepackage{placeins}
\usepackage{gensymb}
\usepackage{xcolor}
\usepackage{listings}
\usepackage{amsmath}
\usepackage{textgreek}
\usepackage{mathtools}
\usepackage[obeyFinal]{easy-todo}
\usepackage[utf8]{inputenc}
\def\LTfontencoding{L7x}
\usepackage[\LTfontencoding]{fontenc}
\usepackage[lithuanian]{babel}
%\usepackage{times}

%\renewcommand{\sfdefault}{uhv}
%\renewcommand{\rmdefault}{utm}
%\renewcommand{\ttdefault}{ucr}

\usepackage{VUMIF}

%Kodo highlitinimo configas
\lstset{basicstyle=\ttfamily,
	showstringspaces=false,
	commentstyle=\color{red},
	keywordstyle=\color{blue}
	}


% Titulinio puslapio reikalai
\vumifdept{Programų sistemų katedra}
\vumifpaper{Bakalaurinis darbas}
\title{Autonominis ketursraigčio skrydžio valdymas\\Autonomus Control of Quadcopter Flight}
\author{
    4 kurso 1 grupės studentas \\
    Rytis Karpuška
}

\supervisor{Irus Grinis, lekt.}
\reviewer{Vytautas Valaitis}
\date{Vilnius \\
	2014}


\begin{document}

%titulinis ir turinys
\maketitle
\tableofcontents



\vumifsectionnonum{Įvadas}



\section{Ketursraigčio techninė įranga}
\subsection{Rėmas, varikliai ir propeleriai}
\subsection{Valdymo elektronika}



\section{Fizikinis modelis}
\subsection{Lokali ir globali koordinačių sistemos}
\subsection{Keliamoji bei sukamoji jėgos}
\subsection{Bendras judėjimo modelis}



\section{Kampinės padėties skaičiavimas}
\subsection{Kvaternionai}
\subsection{Kampinės padėties skaičiavimas pagal giroskopą}
\subsection{Kampinės padėties skaičiavimas pagal akselerometrą}
\subsection{Galutinės kampinės padėties radimas}



\section{Kampinės padėties valdymo algoritmas}
\subsection{PID valdymo algoritmas}
\subsection{PID pritaikymas ketursraigčio valdymui}



\section{Skrydžio valdymas}
\subsection{Atviro-ciklo valdymas}
\subsection{Kampinės pozicijos tikslų lentelė}
\subsection{Atviro-ciklo valdymo trūkumai}



\section{Programinė įranga}
\subsection{Bendroji architektūra}
\subsection{Kompiuteriui skirtas klientas}
\subsection{Retransmitorius}
\subsection{Ketursraigčio pagrindinis valdiklis}

\vumifsectionnonum{Išvados}



\bibliography{Bibliografija}
\begin{itemize}%TODO: fixme
	\item [[AAJ+01]] - \textit{Implementing a Sensor Fusion Algorithm for 3D Orientation Detection with Inertial/Magnetic Sensors}, \url{http://franciscoraulortega.com/pubs/Algo3DFusionsMems.pdf}
	\item [[SSF+11]] - \textit{A sensor fusion algorithm for an integrated angular position estimation with inertial measurement units}, \url{http://www.date-conference.com/proceedings/PAPERS/2011/DATE11/PDFFILES/IP1_06.PDF}
	\item [[MS11]] - \textit{Modeling, Design and Experimental Study for a Quadcopter System Construction}, \url{http://brage.bibsys.no/xmlui/bitstream/id/86811/uiareport.pdf}
\end{itemize}


\end{document}



