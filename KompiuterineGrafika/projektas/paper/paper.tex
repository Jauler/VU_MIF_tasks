\documentclass[12pt, a4paper, lithuanian, final]{article}

\usepackage{hyperref}
\usepackage{graphicx}
\usepackage{float}
\usepackage{placeins}
\usepackage{gensymb}
\usepackage{amsmath}
\usepackage{textgreek}
\usepackage{mathtools}
\usepackage[obeyFinal]{easy-todo}
\usepackage[utf8]{inputenc}
\def\LTfontencoding{L7x}
\usepackage[\LTfontencoding]{fontenc}
\usepackage[lithuanian]{babel}
%\usepackage{times}

%\renewcommand{\sfdefault}{uhv}
%\renewcommand{\rmdefault}{utm}
%\renewcommand{\ttdefault}{ucr}

\usepackage{VUMIF}


% Titulinio puslapio reikalai
\vumifdept{Programų sistemų katedra}
\vumifpaper{Projektinis darbas}
\title{Realaus objektų sukamųjų judesiu perdavimas į virtualią aplinką}
\author{
    4 kurso 1 grupės studentas \\
    Rytis Karpuška
}

\supervisor{R. Krasauskas, doc.}
\date{Vilnius \\
	2014}


\begin{document}

%titulinis ir turinys
\maketitle

\vumifsectionnonum{Įvadas}

Sukamųjų judesių atvaizdavimas kompiuterio trimatėje erdvėje pasinaudojant populiariais pelės ir klaviatūros įeities įrenginiais turi atvaizduoti trimačius judesius per dvimatę ar net vienamtę (klaviatūros atveju) aplinką.
Atkreipiant dėmesį į šią problemą, buvo bandoma sukurti įeities įrenginį, kuris neturi šio apribojimo, ir trimačius judesius atvaizduoja trimatėje erdvėje, panaudojant laisvai prieinamą techninę įrangą.
Kaip įeities įranginys buvo pasirinktas išmanusis telefonas turintis inercinius giroskopo, bei akselerometro sensorius ir magnetometro sensorių.


\section{Kompiliacija bei paleidimas}




\section{Veikimo principai}





\end{document}



