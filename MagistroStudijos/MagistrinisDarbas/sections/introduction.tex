%Įvade aprašomi darbo tikslai, nurodomas temos aktualumas, aptariamos teorinės
%darbo prielaidos bei metodologija, apibrėžiamas tiriamasis objektas,
%apibūdinami su tema susiję literatūros ar kitokie šaltiniai, temos analizės
%tvarka, darbo atlikimo aplinkybės, pateikiama žinių apie naudojamus
%instrumentus (programas ir kt.). Rekomenduojama įvado apimtis 3-4 puslapiai.

\sectionnonum{Įvadas}

% Kas yra biometrines identifikavimo sistemos
Kiekvieno žmogaus kūnas turi aibę požymių, pagal kuriuos jį galima unikaliai identifikuoti (pvz.: pirštų atspaudai).
Šių požymių egzistavimas davė pagrindą {\it biometrinėms identifikavimo sistemoms}.
Šių sistemų paskirtis yra tarp visų užregistruotų žmonių surasti tą, kuriam priklauso duotieji biometriniai požymiai.
Šiame darbe nagrinėjamoje sistemoje \cite{NeurotechnologyMegamatcherAccelerator} sąrašas žmonių, tarp kurių yra vykdoma paieška, yra papildomai atrenkamas pagal vartotojo pateiktą užklausą.
Ši užklausa sąrašą paieškai sudaro pagal papildomus, ne biometrinius, duomenis (pvz.: amžių, lytį, gyvenamąją vietą) priskirtus kiekvienam sistemoje užregistruotam žmogui.


% Kas yra biografinė užklausa ir biografiniai atributai?
\paragraph{Biografiniai atributai bei biografinė schema}

Aptartieji papildomi duomenys vadinami {\it biografiniais atributais}.
Verta atkreipti dėmesį į tai, kad biografiniai atributai yra papildomi, būtinai ne biometriniai, duomenys.

Kiekvienas biografinis atributas turi savo vardą (pvz.: „Miestas“, „Amžius“), bei reikšmę (pvz.: „Vilnius“, „25-eri metai“).
Visų atributų vardų aibė yra vadinama {\it biografine schema}.
Pavyzdžiui \ref{tab:exampleGallery} lentelėje pateikiamame duomenų bazės pavyzdyje biografinė schema būtų \{„Miestas“, „Amžius“\}.

\begin{table}[H]\footnotesize
	\centering
	\begin{tabular}{|c|c|c|c|}
		\hline
		\multirow{2}{*}{{\bf Žmogus}} & \multirow{2}{*}{{\bf Biometriniai požymiai}} & \multicolumn{2}{|c|}{{\bf Biografiniai atributai}}  \\ \cline{3-4}
		& & {\bf Miestas} & {\bf Amžius} \\
		\hline
		Mindaugas & biometrinių požymių įrašas & Vilnius & 35 \\ \cline{2-4}
		\hline
		Petras & biometrinių požymių įrašas & Utena & 15 \\ \cline{2-4}
		\hline
		Eglė & biometrinių požymių įrašas & Zarasai & 10 \\ \cline{2-4}
		\hline
		Dovilė & biometrinių požymių įrašas & Kelmė & 20 \\ \cline{2-4}
		\hline
		Rytis & biometrinių požymių įrašas & Marijampolė & 45 \\ \cline{2-4}
		\hline
		Tomas & biometrinių požymių įrašas & Anykščiai & 30 \\ \cline{2-4}
		\hline
	\end{tabular}
	\caption{Pavyzdiniai biometrinės identifikavimo sistemos duomenys}
	\label{tab:exampleGallery}
\end{table}

Kiekvienam žmogui gali būti priskiriamas daugiau negu vienas biografinis atributas, tačiau visiems žmonėms priskiriamų biografinių atributų schema (biografinė schema) turi būti tokia pati.
Tai reiškia, kad, pavyzdžiui, jeigu Petrui (žr.: \ref{tab:exampleGallery} lentelę) buvo priskirtas gyvenamasis miestas ir amžius, tai visiems likusiems užregistruotiems žmonės irgi bus priskiriamas gyvenamasis miestas ir amžius.

Nagrinėjama sistema palaiko dviejų tipų biografinius atributus:
\begin{itemize}
\item skaitinio tipo
\item simbolių eilutės tipo
\end{itemize}
Tačiau esant poreikiui šis atributų tipų sąrašas gali būti plečiamas.
Sistemoje visi tą patį vardą turintys atributai yra ir to pačio tipo.






\paragraph{Duomenų bazės modelis}

Šiame darbe remiamasi duomenų bazės modeliu „Data cube“ \cite{marcel2000modeling}.
Nagrinėjamos sistemos atveju, kiekvienas atributas atitinka vieną, konkrečią, daugiamatės erdvės dimensiją.

\begin{figure}[H]
\begin{center}
\includegraphics[width=0.5\textwidth]{img/MultidimensionalGallery.png}
\caption{Pavyzdiniai biografinių atributų rinkiniai dvimatėje erdvėje.}
\label{img:multidimensionalGallery}
\end{center}
\end{figure}

\ref{img:multidimensionalGallery} paveikslėlyje pateikiamas tokios erdvės pavyzdys atitinkantis lentelėje \ref{tab:exampleGallery} pateiktą duomenų bazės pavyzdį.

Duomenų bazės įrašams apdoroti bus naudojama palyginimo funkciją (\ref{eq:threeWayComparisonFunction}).
\begin{equation}
	f_A(x_1, x_2)=
\begin{cases}
	-1,& \text{jeigu } x_1 < x_2\\
	0,& \text{jeigu } x_1 = x_2\\
	1,& \text{jeigu } x_1 > x_2\\
\end{cases}
\label{eq:threeWayComparisonFunction}
\end{equation}

Čia $x_1$ ir $x_2$ yra tą patį vardą $A$ turinčių atributų reikšmės.
Simbolių eilutės tipo atributų reikšmės gali būti lyginamos leksikografine tvarka.

Šiame darbe daroma prielaida, kad palyginimo funkcijos (\ref{eq:threeWayComparisonFunction}) atžvilgiu atributo reikšmių aibė yra pilnai sutvarkyta \cite{hrbacek1999introduction}.




% Biometrinių įrašų blokai - particijos daugiamatėje erdvėje
\paragraph{Biometrinių įrašų blokai daugiamatėje erdvėje}

Nagrinėjamos sistemos bazinis elementas yra įrašas.
Vieną įrašą sudaro biometrinių požymių rinkinys susietas su biografinių atributų reiškmių rinkiniu.
Sistemoje, operatyviojoje atmintyje, įrašai yra saugomi ir apdorojami ne po vieną, bet grupėmis.
Šios grupės yra vadinamos {\it biometrinių įrašų blokais}.
Kiekvienam įrašui registracijos metu yra laisvai parenkamas vienas blokas, kuriame jis bus saugomas.

Šiame darbe nagrinėjamame duomenų bazės modelyje biometrinių įrašų blokai yra ankščiau aptartos daugiamatės erdvės sritys, kurios apima visus konkrečiam blokui priskirtus įrašus (žr.: \ref{img:multidimensionalPartitionedGallery} pav.)

\begin{figure}[H]
\begin{center}
\includegraphics[width=0.5\textwidth]{img/MultidimensionalPartitionedGallery.png}
\caption{B1-B4 - erdvės sritys atitinkančios biometrinių įrašų blokus}
\label{img:multidimensionalPartitionedGallery}
\end{center}
\end{figure}



\paragraph{Biografinės užklausos}

Nurodžius pasirinktų biografinių atributų kitimo sritis -- apibrėžiama įrašų aibė, kurioje ieškomi pasirinktų biometrinių požymių atitikmenys.
Tokios paieškos užduotis vadinama {\it biografine užklausa}.
Šiai užklausai aprašyti yra skirta gramatika, kurios Backus ir Nauro forma \cite{mccracken2003backus} yra pateikiama \ref{tab:queryBNF} lentelėje.

\begin{table}[H]\footnotesize
	\centering
	\begin{tabular}{|l c l|}
		\hline
		užklausa                      & ::= & <vienaris operatorius> <operandas> | \\
									  &     & \multicolumn{1}{l|}{<operandas> <dvinaris operatorius> <operandas> |} \\
									  &     & \multicolumn{1}{l|}{"(" <užklausa> ")" |} \\
									  &     & \multicolumn{1}{l|}{ <atributo vardas> <sąrašo operatorius> <sąrašas>} \\
		operandas                     & ::= & <užklausa> | <atributo vardas> | "'" <skaičius> "'" | "'" <žodis> "'" \\
		vienaris operatorius          & ::= & <neprivalomas tarpas> "NOT" <neprivalomas tarpas> \\
		dvinaris operatorius          & ::= & <neprivalomas tarpas> <dvinario operatoriaus ženklas> <neprivalomas tarpas> \\
		dvinario operatoriaus ženklas & ::= & ">" | ">=" | "<" | "<=" | "=" | "<>" | "AND" | "OR" \\
		sąrašo operatorius            & ::= & <neprivalomas tarpas> "IN" <neprivalomas tarpas> \\
		sąrašas                       & ::= & "(" <sąrašo elementai> ")" \\
		sąrašo elementai              & ::= & <sąrašo elementas> | <sąrašo elementas> "," <sąrašo elementai> \\
		sąrašo elementas              & ::= & "'" <žodis> "'" | "'" <skaičius> "'" \\
		atributo vardas               & ::= & <neprivalomas tarpas> <žodis> <neprivalomas tarpas> \\
		neprivalomas tarpas           & ::= & "" | " " <neprivalomas tarpas> \\
		žodis                         & ::= & <raidė> | <žodis> <raidė> | <žodis> <skaičius> \\
		skaičius                      & ::= & "0" | "1" | "2" | "3" | "4" | "5" | "6" | "7" | "8" | "9" \\
		raidė                         & ::= & "A" | "B" | "C" ... "Z" | "a" | "b" | "c" ... "z"  \\
		\hline
	\end{tabular}
	\caption{Biografinės užklausos aprašymo gramatikos Backus ir Nauro forma}
	\label{tab:queryBNF}
\end{table}

Keletas užklausos pavyzdžių pateikiama \ref{tab:queryExamples} lentelėje laikant, kad biografinė schema yra \{„Miestas“, „Amžius“\}.

\begin{table}[H]\footnotesize
	\centering
	\begin{tabular}{|c|c|l|}
		\hline
		& {\bf Užklausos aprašymas} & {\bf Užklausos interpretacija} \\
		\hline
		Pavyzdys 1 & Amžius >= '18' & Visi suaugę žmonės\\
		\hline
		Pavyzdys 2 & Amžius >= '18' AND Miestas IN ('Vilnius', 'Kaunas') & Visi suagę vilniečiai ir kauniečiai\\
		\hline
		Pavyzdys 3 & NOT (Miestas = 'Vilnius' AND Amžius >= '18') & Visi žmonės išskyrus suaugusius vilniečius\\
		\hline
	\end{tabular}
	\caption{Užklausų aprašymo pavyzdžiai.}
	\label{tab:queryExamples}
\end{table}



\paragraph{Biografinių užklausų apdorojimas}

Smulkiausias įrašų sąrašas kuriame sistema \cite{NeurotechnologyMegamatcherAccelerator} šiuo metu gali vykdyti paiešką yra vienas biometrinių įrašų blokas.
Todėl vartotojo biografinės užklausos apdorojimas yra skaidomas į tokius tris etapus:
\begin{enumerate}
	\item Atmetimo etapas: Atmetami visi blokai, neturintys įrašų, kurių atributų rinkiniai atitinka vartotojo užklausą.
	\item Paieškos etapas: Atliekama paieška blokuose, kurie liko po atmetimo etapo.
	\item Filtravimo etapas: Atmetami visi paieškos etape rasti įrašai, kurių atributų rinkiniai neatitinka vartotojo užklausos.
\end{enumerate}

Verta pastebėti, kad po atmetimo etapo likęs blokų kiekis priklauso nuo metodo, pagal kurį yra parenkama daugiamatės erdvės sritis atributų rinkinių saugojimui.



\paragraph{Darbo tikslas}

Šio darbo metu siekiama padidinti biometrinės identifikavimo sistemos \cite{NeurotechnologyMegamatcherAccelerator} pralaidumą (užklausų skaičių per laiko vienetą).
Tuo tikslu ketinama minimizuoti biometrinių įrašų blokų skaičių, kuris lieka po atmetimo etapo.

Tam bus pritaikyti ir palyginti du įrašų priskirimo blokams metodai:
\begin{itemize}
		\item Metodas paremtas priešdėliniu rūšavimu.
		\item Metodas paremtas K-d medžiu.
\end{itemize}

Darbą planuojama skirstyti į tokius uždavinius:

\begin{enumerate}
	\item Išanalizuoti tipinių užklausų statistines savybes, pasiruošti testavimo duomenis.
	\item Empiriškai įvertinti sistemos pralaidumo priklausomybę nuo įrašų skaičiaus bloke.
	\item Įdiegti metodą paremtą priešdėliniu rūšiavimu.
	\item Įdiegti metodą paremtą K-d medžiu.
	\item Palyginti įdiegtus metodus tarpusavyje ir su sistema \cite{NeurotechnologyMegamatcherAccelerator}
\end{enumerate}

Numatomas darbų eiliškumas:
\begin{enumerate}
	\item Apžvelgiami kiti panašūs darbai bei pateikiami šaltiniai, kuriais remiasi šis darbas.
	\item Apžvelgiama kokios užklausų savybės išplaukia iš apibrėžtos gramatikos, kokiomis statistinėmis savybėmis pasižymi vartotojų dažniausiai pateikiamos užklausos.
	\item Aprašoma kaip bus palyginami įrašų priskyrimo sritims metodai, bei aprašomas testavimo duomenų rinkinys.
	\item Empiriškai nustatoma sistemos pralaidumo priklausomybė nuo maksimalaus bloko dydžio. Parenkamas bloko dydis, kuris bus naudojamas palyginant metodus.
	\item Įgyvendinamas ir palyginamas metodas paremtas priešdėliniu rūšiavimu.
	\item Įgyvendinamas ir palyginamas metodas paremtas K-d medžiu.
	\item Aprašomi pasiekti rezultatai bei pateikiamos išvados.
\end{enumerate}


