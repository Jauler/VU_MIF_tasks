%Įvade aprašomi darbo tikslai, nurodomas temos aktualumas, aptariamos teorinės
%darbo prielaidos bei metodologija, apibrėžiamas tiriamasis objektas,
%apibūdinami su tema susiję literatūros ar kitokie šaltiniai, temos analizės
%tvarka, darbo atlikimo aplinkybės, pateikiama žinių apie naudojamus
%instrumentus (programas ir kt.). Rekomenduojama įvado apimtis 3-4 puslapiai.

\sectionnonum{Įvadas}

% abstract intro
Šių dienų visuomenėje galimybė patikimai identifikuoti asmens tapatybę yra labai svarbi kriminalistikos, krašto apsaugos, balsavimo, finansų, internetinio saugumo ir kitose srityse.
Dažnai identifikavimas yra atliekamas pagal tai ką asmuo turi (pvz.: pasas, ID kortelė), ką asmuo žino (pvz.: slaptažodis) arba pagal asmens kūno požymius (pvz.: pirštų atspaudai).

Sistemos padedančios automatiškai identifikuoti žmogų pagal jo kūno požymius (Toliau: biometrinius požymius) vadinamos {\it biometrinėmis identifikavimo sistemomis}.
Dažnas tokios sistemos taikymas yra labai didelio mąsto ir sistemos dažnai atlieka milijonus ar net milijardus biometrinių požymių palyginimų per sekundę, pavyzdžiui naudojant oro uostuose \cite{ForbesAirportBiometrics}, migracijos departamentuose \cite{NeurotechnologySomaliNationalId}, rinkimų komisijose \cite{NeurotechnologyDRCDedublicationProject} \cite{NeurotechnologyVenezuelaDedublicationProject} ir kitur.
Tai didina sistemų kainą bei kompleksiškumą.

Verta pastebėti, kad dažnai be biometrinių požymių yra žinomi asmens metaduomenys (pvz.: amžius, lytis, gyvenamoji vieta ir kt.) pagal kuriuos galima stipriai susiaurinti paieškos ratą.

Šiame darbe siekiama susiaurinti ir tuo pačiu pagreitinti, biometrinę identifikaciją panaudojant asmens metaduomenis.
Darbe nagrinėjama biometrinio identifikavimo sistema \cite{NeurotechnologyMegamatcherAccelerator}.


% Kas yra biografinė užklausa ir biografiniai atributai?
\paragraph{Biografiniai atributai}

Aptartieji metaduomenys vadinami {\it biografiniais atributais}.
Verta atkreipti dėmesį į tai, kad biografiniai atributai yra papildomi, būtinai ne biometriniai, duomenys.

Šiame darbe laikoma, kad kiekvienas biografinis atributas turi savo vardą (pvz.: „Miestas“, „Amžius“), bei reikšmę (pvz.: „Vilnius“, „25-eri metai“).
Taip pat kiekvienam asmeniui gali būti priskiriamas daugiau negu vienas atributas (pvz.: „Miestas“ ir „Amžius“).

\begin{table}[H]\footnotesize
	\centering
	\begin{tabular}{|c|c|c|c|}
		\hline
		\multirow{2}{*}{{\bf Asmuo}} & \multirow{2}{*}{{\bf Biometriniai požymiai}} & \multicolumn{2}{|c|}{{\bf Biografiniai atributai}}  \\ \cline{3-4}
		& & {\bf Miestas} & {\bf Amžius} \\
		\hline
		Mindaugas & biometrinių požymių įrašas & Vilnius & 35 \\ \cline{2-4}
		\hline
		Petras & biometrinių požymių įrašas & Utena & 15 \\ \cline{2-4}
		\hline
		Eglė & biometrinių požymių įrašas & Zarasai & 10 \\ \cline{2-4}
		\hline
		Dovilė & biometrinių požymių įrašas & Kelmė & 20 \\ \cline{2-4}
		\hline
		Rytis & biometrinių požymių įrašas & Marijampolė & 45 \\ \cline{2-4}
		\hline
		Tomas & biometrinių požymių įrašas & Anykščiai & 30 \\ \cline{2-4}
		\hline
	\end{tabular}
	\caption{Pavyzdiniai biometrinės identifikavimo sistemos duomenys}
	\label{tab:exampleGallery}
\end{table}

Lentelėje \ref{tab:exampleGallery} pateikiamas duomenų bazės pavyzdys.


% Biometrinių požymių blokai - particijos daugiamatėje erdvėje
\paragraph{Biometrinių požymių blokai}

Siekiant išnaudoti spartinančiosios atminties privalumus, nagrinėjamoje sistemoje biometriniai požymiai yra apdorojami ne po vieną, bet grupėmis.
Požymiai į grupes gali būti skirstomi laisvai, tačiau tik vieną kartą (priskyrus asmens $p_i$ biometrinius požymius $b_j$ į grupę $B_k$ veliau jau nebegalima šio požymio priskirti kitai grupei $B_l$).
Šios grupės vadinamos {\it biometrinių požymių blokais}.

Taigi vienas biometrinių požymių blokas saugo aibę biometrinių požymių $\{b_1, b_2, ..., b_n\}$, kurie atitinkamai priklauso asmenims $\{p_1, p_2, ..., p_n\}$, o su kiekvienu asmeniu yra susieti briografiniai atributai $\{a_a, a_b, ..., a_z\}$, todėl blokams yra priskirta ir biografinių atributų matrica:
\begin{equation}
\begin{bmatrix}
		a_{1a}       & a_{2a} & \dots & a_{3a} \\
		a_{1b}       & a_{2b} & \dots & a_{3b} \\
		\hdotsfor{4} \\
		a_{1z}       & a_{2z} & \dots & a_{3z}
	\end{bmatrix}
\end{equation}


\paragraph{Biografinės užklausos}

Nurodžius pasirinktų biografinių atributų kitimo sritis -- apibrėžiama asmenų, ir jų biometrinių požymių, aibė, kurioje vykdoma identifikacija.
Tokios identifikacijos užduotis vadinama {\it biografine užklausa}.
Šiai užklausai aprašyti yra skirta gramatika, kurios Backus ir Nauro forma \cite{mccracken2003backus} yra pateikiama \ref{tab:queryBNF} lentelėje.

\begin{table}[H]\footnotesize
	\centering
	\begin{tabular}{|l c l|}
		\hline
		užklausa                      & ::= & <vienaris operatorius> <operandas> | \\
									  &     & \multicolumn{1}{l|}{<operandas> <dvinaris operatorius> <operandas> |} \\
									  &     & \multicolumn{1}{l|}{"(" <užklausa> ")" |} \\
									  &     & \multicolumn{1}{l|}{ <atributo vardas> <sąrašo operatorius> <sąrašas>} \\
		operandas                     & ::= & <užklausa> | <atributo vardas> | "'" <skaičius> "'" | "'" <žodis> "'" \\
		vienaris operatorius          & ::= & <neprivalomas tarpas> "NOT" <neprivalomas tarpas> \\
		dvinaris operatorius          & ::= & <neprivalomas tarpas> <dvinario operatoriaus ženklas> <neprivalomas tarpas> \\
		dvinario operatoriaus ženklas & ::= & ">" | ">=" | "<" | "<=" | "=" | "<>" | "AND" | "OR" \\
		sąrašo operatorius            & ::= & <neprivalomas tarpas> "IN" <neprivalomas tarpas> \\
		sąrašas                       & ::= & "(" <sąrašo elementai> ")" \\
		sąrašo elementai              & ::= & <sąrašo elementas> | <sąrašo elementas> "," <sąrašo elementai> \\
		sąrašo elementas              & ::= & "'" <žodis> "'" | "'" <skaičius> "'" \\
		atributo vardas               & ::= & <neprivalomas tarpas> <žodis> <neprivalomas tarpas> \\
		neprivalomas tarpas           & ::= & "" | " " <neprivalomas tarpas> \\
		žodis                         & ::= & <raidė> | <žodis> <raidė> | <žodis> <skaičius> \\
		skaičius                      & ::= & "0" | "1" | "2" | "3" | "4" | "5" | "6" | "7" | "8" | "9" \\
		raidė                         & ::= & "A" | "B" | "C" ... "Z" | "a" | "b" | "c" ... "z"  \\
		\hline
	\end{tabular}
	\caption{Biografinės užklausos aprašymo gramatikos Backus ir Nauro forma}
	\label{tab:queryBNF}
\end{table}

Keletas užklausos pavyzdžių pateikiama \ref{tab:queryExamples} lentelėje laikant, kad asmenims priskirti biografiniai atributai yra „Miestas“ ir „Amžius“.

\begin{table}[H]\footnotesize
	\centering
	\begin{tabular}{|c|c|l|}
		\hline
		& {\bf Užklausos aprašymas} & {\bf Užklausos interpretacija} \\
		\hline
		Pavyzdys 1 & Amžius >= '18' & Visi suaugę žmonės\\
		\hline
		Pavyzdys 2 & Amžius >= '18' AND Miestas IN ('Vilnius', 'Kaunas') & Visi suagę vilniečiai ir kauniečiai\\
		\hline
		Pavyzdys 3 & NOT (Miestas = 'Vilnius' AND Amžius >= '18') & Visi žmonės išskyrus suaugusius vilniečius\\
		\hline
	\end{tabular}
	\caption{Užklausų aprašymo pavyzdžiai.}
	\label{tab:queryExamples}
\end{table}



\paragraph{Biografinių užklausų apdorojimas}

Smulkiausias biometrinių požymių sąrašas kuriame sistema \cite{NeurotechnologyMegamatcherAccelerator} gali vykdyti identifikacija yra vienas biometrinių požymių blokas.
Todėl biografinės užklausos apdorojimas yra skaidomas į tokius tris etapus:
\begin{enumerate}
	\item Atmetimo etapas: Atmetami visi blokai, neturintys asmenų, kurių atributų rinkiniai atitinka vartotojo užklausą.
	\item Identifikacijos etapas: Atliekama identifikacija blokuose, kurie liko po atmetimo etapo.
	\item Filtravimo etapas: Atmetami visi identifikacijos etape rasti asmenys, kurių atributų rinkiniai neatitinka vartotojo užklausos.
\end{enumerate}

Verta pastebėti, kad po atmetimo etapo likęs blokų (ir atitinkmai biometrinių požymių) kiekis priklauso nuo metodo, pagal kurį yra parenkama kuriam blokui bus priskiriami asmens biometriniai požymiai bei biografiniai atributai.

Šio darbo metu bus nagrinėjami du požymių priskyrimo blokas metodai:
\begin{itemize}
		\item Metodas paremtas priešdėliniu rūšavimu.
		\item Metodas paremtas K-d medžiu.
\end{itemize}

\paragraph{Uždaviniai}
Darbą planuojama skirstyti į tokius uždavinius:

\begin{enumerate}
	\item Išanalizuoti tipinių užklausų statistines savybes, pasiruošti testavimo duomenis.
	\item Empiriškai įvertinti sistemos pralaidumo priklausomybę nuo požymių skaičiaus bloke.
	\item Įdiegti metodą paremtą priešdėliniu rūšiavimu.
	\item Įdiegti metodą paremtą K-d medžiu.
	\item Palyginti įdiegtus metodus tarpusavyje ir su sistema \cite{NeurotechnologyMegamatcherAccelerator}
\end{enumerate}

Numatomas darbų eiliškumas:
\begin{enumerate}
	\item Apžvelgiami kiti panašūs darbai bei pateikiami šaltiniai, kuriais remiasi šis darbas.
	\item Apžvelgiama kokios užklausų savybės išplaukia iš apibrėžtos gramatikos, kokiomis statistinėmis savybėmis pasižymi vartotojų dažniausiai pateikiamos užklausos.
	\item Aprašoma kaip bus palyginami požymių priskyrimo sritims metodai, bei aprašomas testavimo duomenų rinkinys.
	\item Empiriškai nustatoma sistemos pralaidumo priklausomybė nuo maksimalaus bloko dydžio. Parenkamas bloko dydis, kuris bus naudojamas palyginant metodus.
	\item Įgyvendinamas ir palyginamas metodas paremtas priešdėliniu rūšiavimu.
	\item Įgyvendinamas ir palyginamas metodas paremtas K-d medžiu.
	\item Aprašomi pasiekti rezultatai bei pateikiamos išvados.
\end{enumerate}


