\section{Literatūros apžvalga}

Duomenų bazėse neretai kyla panaši problema į aptartą įvade.
Čia minimizuojamas disko operacijų skaičius siekiant sumažinti duomenų bazės atsako laiką bei padidinti pralaidumą (apdorotų užklausų skaičių per laiko veinetą) \cite{garcia2000database}.
Tuo tikslu yra naudojama duomenų struktūra, vadinama {\it duomenų indeksu}.
Jeigu duomenys yra daugiamačiai (vienas įrašas gali susidėti iš daugiau negu vienos reikšmės), tuomet indeksas tokiems duomenims vadinamas {\it daugiamačių duomenų indeksu}, o metodas pagal kurį šis indeksas yra sudaromas, bei naudojamas {\it daugiamačių duomenų indeksavimo metodu}.

% TODO: some article which seperates point queries and range queries

Lu ir Ooi \cite{lu1993spatial} bei Gaede ir Günther \cite{gaede1998multidimensional} detaliai apžvelgia įvairius daugiamačių duomenų indeksavimo metodus, bei pateikia schemą padedančią suprasti šių metodų istoriją (žiūrėti priedą \ref{app:multidimensionalIndexing}).

Autoriai šiuos metodus suskirsto į šias tris grupes:
\begin{itemize}
	\item Metodai paremti maišos funkcijomis
	\item Hierarchiniai metodai
	\item Metodai paremti erdvę užpildančiomis kreivėmis \cite{bader2012space}
\end{itemize}

Verta pastebėti, kad dažnai maišos funkcijomis paremti metodai nėra tinkami reikšmių intervalo užklausoms \cite{nievergelt1981grid} \cite{tamminen1982excell}. % TODO: explain what is "reikšmių intervalo užklausos"
Todėl šiame darbe tokie metodai nebus įgyvendinami ir lyginami sistemoje \cite{NeurotechnologyMegamatcherAccelerator}.


\subsection{Maišos funkcijomis paremti metodai daugiamačių duomenų indeksavimui}
% generic description
% basically hash tables where buckets are bloks in multidimensional space
% order preserving hashes
% pritaikymo pavyzdžiai


\subsection{Hierarchiniai metodai daugiamačių duomenų indeksavimui}
% generic description
% split strategies (predetermined, data based, distribution based)
% pritaikymo pavyzdžiai


\subsection{Erdvę užpildančiomis kreivėmis paremti metodai daugiamačių duomenų indeksavimui}
% generic description
% row-wise kreivė, panaši į šiame darbe testuojamą metodą
% kitos kreivės
% pritaikymo pavyzdžiai


\subsection{SP-GIST karkasas}
% sp-gists
% bulk operations on sp-gists
% postgre (min 3)
