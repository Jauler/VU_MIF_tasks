\section{Literatūros apžvalga}

Panaši problema į aptartąją įvade yra nemažai tyrinėjama duomenų bazių kontekste.
Čia minimizuojamas disko operacijų skaičius siekiant sumažinti duomenų bazės atsako laiką bei padidinti pralaidumą (apdorotų užklausų skaičių per laiko vienetą) \cite{garcia2000database}.
Panašiai kaip šiame darbe siekiama minimizuoti biometrinių įrašų blokų skaičių likusį po atmetimo etapo siekiant pagerinti sistemos \cite{NeurotechnologyMegamatcherAccelerator} pralaidumą.
Tuo tikslu yra naudojama duomenų struktūra, vadinama {\it duomenų indeksu}.
Jeigu duomenys yra daugiamačiai (vienas įrašas gali susidėti iš daugiau negu vienos reikšmės), tuomet indeksas tokiems duomenims vadinamas {\it daugiamačių duomenų indeksu}, o metodas pagal kurį šis indeksas yra sudaromas bei naudojamas {\it daugiamačių duomenų indeksavimo metodu}.

\subsection{Užklausų klasifikacija}

Šiame darbe palyginant indeksavimo metodus sistemoje \cite{NeurotechnologyMegamatcherAccelerator} siekiama apimti kiek galima daugiau užklausų klasių pagal Gaede ir Günther \cite{gaede1998multidimensional} pateikiamą daugiamačių užklausų klasifikaciją:
\begin{enumerate}
	\item Griežto atitikmens užklausa
	\item Taško užklausa
	\item Lango užklausa
	\item Regiono užklausa
	\item Apgaubiančioji užklausa
	\item Pilno regiono užklausa
	\item Kaimynų užklausa
	\item Artimiausio kaimyno užklausa
\end{enumerate}

% TODO: mention that this is finite space and that some operations like and = and similar are defined (and how defined)
Autorius duomenis ir užklausas nagrinėja $d$ dimensijų euklido erdvėje $E^d$.
Duomenys, šioje erdvėje, gali būti nebūtinai taškai, bet bet kokia erdvės sritis (taškas, linija, kvadratas, trapecija, kubas, piramidė...).
Šios erdvės sritys užrašomas $o.G$, kur $o$ yra duomuo (kartais vadinamas objektu), o $o.G$ yra erdvės sritis priskiriama tam objektui/duomeniui.

\subsubsection{Griežto atitikmens užklausa}
Griežto atitikmens užklausa yra tokia užklausa, kuri duotai erdvės sričiai $g$ erdvėje $E^d$ randa visus objektus, kurių erdvės sritys sutampa su duotąja:

\begin{equation}
	EMQ(g) = \{ o | g = o.G \}
\label{eq:ExactMatchQuery}
\end{equation}

\subsubsection{Taško užklausa}
Taško užklausa yra tokia užklausa, kuri duotam taškui $p$ erdvėje $E^d$ randa visus objektus, kurie turi bendrų taškų:

\begin{equation}
	PQ(p) = \{ o | p \cap o.G = p \}
\label{eq:ExactMatchQuery}
\end{equation}

Čia $\cap$ reiškia dviejų sričių sankirta euklido erdvėje $E^d$, t.y. visi bendri taškai, kuriuos apima abi sritys.
Žiūrėti priedą \ref{app:pointQuery}.

\subsubsection{Lango užklausa}
Lango užklausa, tai tokia užkausa, kuri duotiems $d$ intervalams $I^d = [l_1, u_2] \times [l_2, u_2] \times ... \times [l_d, u_d]$ (po vieną kiekvienai erdvės $E^d$ koordinačių ašiai) randa visus objektus $o$ kurie turi bendrų taškų:

\begin{equation}
	WQ(I^d) = \{ o | I^d \cap o.G \neq \emptyset \}
\label{eq:ExactMatchQuery}
\end{equation}

Verta pastebėti, kad lango užklausos atveju visos srities kraštinės yra lygegriačios erdvės $E^d$ koordinačių ašims.
Žiūrėti priedą \ref{app:windowQuery}.


\subsubsection{Regiono užklausa}
Regiono užklausa yra labai panaši į Lango užklausą, tačiau srities kraštinės gali būti laisvai pasirinktos, ir neturi būti lygegriačios koordinačių ašims.
Regiono užklausa randa visus objektus, kurie turi bendrų taškų su duotąja sritimi $g$.

\begin{equation}
	IQ(g) = \{ o | g \cap o.G \neq \emptyset \}
\label{eq:ExactMatchQuery}
\end{equation}
Žiūrėti priedą \ref{app:intersectionQuery}.

\subsubsection{Apgaubiančioji užklausa}
Apgaubiančioji užklausa yra tokia užklausa, kuri randa visus objektus $o$, kurių erdvės sritis $o.G$ pilnai apima pateiktąją erdvės sritį $g$:

\begin{equation}
	EQ(g) = \{ o | (g \cap o.G) = o.G \}
\label{eq:ExactMatchQuery}
\end{equation}
Žiūrėti priedą \ref{app:enclosureQuery}.


\subsubsection{Pilno regiono užklausa}
Pilno regiono užklausa iš esmės yra priešinga apgaubiančiąjai.
Pilno regiono užklausa yra tokia užklausa, kuri randa visus objektus $o$, kurių erdvės sritis $o.G$ pilnai patenka į pateiktąją erdvės sritį $g$:

\begin{equation}
	CQ(g) = \{ o | (g \cap o.G) = g \}
\label{eq:ExactMatchQuery}
\end{equation}
Žiūrėti priedą \ref{app:containmentQuery}.


\subsubsection{Kaimynų užklausa}
Kaimynų užklausa yra tokia užklausa, kuri pagal duotąjį objektą $o'$ suranda visus šio objekto kaimynus, t.y. objektus $o$ kurie turi bendrą kraštinę erdvėje $E^d$:

\begin{equation}
	AQ(g) = \{ o | (o'.G \cap o.G) \neq \emptyset \land o'.G\textdegree \cap o.G\textdegree = \emptyset \}
\label{eq:ExactMatchQuery}
\end{equation}

Čia $o.G\textdegree$ reiškia visus vidinius taškus, kurie priklauso sričiai $o.G$.
Žiūrėti priedą \ref{app:adjacencyQuery}.


\subsubsection{Artimiausio kaimyno užklausa}
Artimiausio kaimyno užklausa yra tokia užklausa, kuri duotam objektui $o'$ randa artimiausią kaimyną (verta pastebėti, kad šiuo atveju kaimynai gali ir neturėti bendrų kraštinių).

\begin{equation}
	NNQ(o') = \{ o | \forall o'' : dist(o'.G, o.G) \leq dist(o'.G, o''.G) \}
\label{eq:ExactMatchQuery}
\end{equation}




\subsection{Daugiamačių duomenų indeksavimo metodų klasifikacija}
Lu ir Ooi \cite{lu1993spatial} bei \cite{gaede1998multidimensional} apžvelgia įvairius daugiamačių duomenų indeksus, bei pateikia schemą padedančią suprasti jų istoriją (žiūrėti priedą \ref{app:multidimensionalIndexing}).

Abu autoriai šiuos metodus suskirsto į tris grupes:
\begin{itemize}
	\item Metodai paremti maišos funkcijomis.
	\item Hierarchiniai metodai.
	\item Metodai paremti erdvę užpildančiomis kreivėmis \cite{bader2012space}.
\end{itemize}



\subsection{Maišos funkcijomis paremti metodai daugiamačių duomenų indeksavimui}

% generic description
% basically hash tables where buckets are bloks in multidimensional space
% order preserving hashes
% pritaikymo pavyzdžiai

Verta pastebėti, kad dažnai maišos funkcijomis paremti metodai yra tinkami tik griežto atitikmens užklausoms \cite{nievergelt1981grid} \cite{tamminen1982excell}.
Todėl šiame darbe tokie metodai nebus įgyvendinami ir lyginami sistemoje \cite{NeurotechnologyMegamatcherAccelerator}.


\subsection{Hierarchiniai metodai daugiamačių duomenų indeksavimui}
% generic description
% split strategies (predetermined, data based, distribution based)
% pritaikymo pavyzdžiai


\subsection{Erdvę užpildančiomis kreivėmis paremti metodai daugiamačių duomenų indeksavimui}
% generic description
% row-wise kreivė, panaši į šiame darbe testuojamą metodą
% kitos kreivės
% pritaikymo pavyzdžiai


\subsection{SP-GIST karkasas}
% sp-gists
% bulk operations on sp-gists
% postgre (min 3)
