\section{Literatūros apžvalga}

Panaši problema į aptartąją įvade yra nemažai tyrinėjama duomenų bazių kontekste.
Čia minimizuojamas disko operacijų skaičius siekiant sumažinti duomenų bazės atsako laiką bei padidinti pralaidumą (apdorotų užklausų skaičių per laiko vienetą) \cite{garcia2000database}.
Panašiai kaip šiame darbe siekiama minimizuoti biometrinių įrašų blokų skaičių likusį po atmetimo etapo siekiant pagerinti sistemos \cite{NeurotechnologyMegamatcherAccelerator} pralaidumą.
Tuo tikslu yra naudojama duomenų struktūra, vadinama {\it duomenų indeksu}.
Jeigu duomenys yra daugiamačiai (vienas įrašas gali susidėti iš daugiau negu vienos reikšmės), tuomet indeksas tokiems duomenims vadinamas {\it daugiamačių duomenų indeksu}, o metodas pagal kurį šis indeksas yra sudaromas bei naudojamas {\it daugiamačių duomenų indeksavimo metodu}.

\subsection{Užklausų klasifikacija}

Šiame darbe palyginant indeksavimo metodus sistemoje \cite{NeurotechnologyMegamatcherAccelerator} siekiama apimti kiek galima daugiau užklausų klasių pagal Gaede ir Günther \cite{gaede1998multidimensional} pateikiamą daugiamačių užklausų klasifikaciją:
\begin{enumerate}
	\item Griežto atitikmens užklausa
	\item Taško užklausa
	\item Lango užklausa
	\item Regiono užklausa
	\item Apgaubiančioji užklausa
	\item Pilno regiono užklausa
	\item Kaimynų užklausa
	\item Artimiausio kaimyno užklausa
\end{enumerate}

Autorius duomenis ir užklausas nagrinėja $d$ dimensijų euklido erdvėje $E^d$.
Atskirus įrašus apibrėžia kaip objektus $o$ kurie gali turėti 0 ar daugiau papildomų atributų nesusijusių su erdve $E^d$ (pvz.: vardas, pavadinimas, amžius...), bei griežtai vieną atributą $o.G$, kuris apibūdina objekto $o$ padėtį erdvėje $E^d$.
Šis $o.G$ atributas yra aibė taškų, kuriuos objektas $o$ užima erdvėje $E^d$.
Objektams yra apibrėžiami operatoriai $=$, $\cap$, bei $dist(o_1, o_2)$.
Du objektai $o_1$ ir $o_2$ skaitomi lygiais $o_1 = o_2$ tada ir tik tada jeigu abiejų objektų užimamų taškų aibės $o_1.G$ ir $o_2.G$ sutampa.
Dviejų objektų $o_1$ ir $o_2$ sankirta $o_1 \cap o_2$ yra aibė taškų, kurie patenka ir į $o_1.G$ ir į $o_2.G$.
Ir galiausiai atstumas tarp dviejų objektų $o_1$ ir $o_2$ $dist(o_1, o_2)$ skaitomas mažiausias atstumas tarp bet kurių dviejų taškų esančių $o_1.G$ ir $o_2.G$ aibėse.



\subsubsection{Griežto atitikmens užklausa}
Griežto atitikmens užklausa yra tokia užklausa, kuri duotąjam objektui $o'$ erdvėje $E^d$ randa visus objektus, kurių erdvės sritys sutampa su duotąja:

\begin{equation}
	EMQ(g) = \{ o | o'.G = o.G \}
\label{eq:ExactMatchQuery}
\end{equation}

\subsubsection{Taško užklausa}
Taško užklausa yra tokia užklausa, kuri duotam taškui $p$ erdvėje $E^d$ randa visus objektus, kurie turi bendrų taškų:

\begin{equation}
	PQ(p) = \{ o | \{p\} \cap o.G = \{p\} \}
\label{eq:ExactMatchQuery}
\end{equation}

Žiūrėti priedą \ref{app:pointQuery}.

\subsubsection{Lango užklausa}
Lango užklausa, tai tokia užkausa, kuri duotiems $d$ intervalams $I^d = [l_1, u_2] \times [l_2, u_2] \times ... \times [l_d, u_d]$ (po vieną kiekvienai erdvės $E^d$ koordinačių ašiai) randa visus objektus $o$ kurie turi bendrų taškų:

\begin{equation}
	WQ(I^d) = \{ o | I^d \cap o.G \neq \emptyset \}
\label{eq:ExactMatchQuery}
\end{equation}

Verta pastebėti, kad lango užklausos atveju visos erdvės srities kraštinės yra lygegriačios erdvės $E^d$ koordinačių ašims.
Žiūrėti priedą \ref{app:windowQuery}.


\subsubsection{Regiono užklausa}
Regiono užklausa yra labai panaši į Lango užklausą, tačiau srities kraštinės gali būti laisvai pasirinktos, ir neturi būti lygegriačios koordinačių ašims.
Regiono užklausa randa visus objektus, kurie turi bendrų taškų su duotuoju objektu $o'$.

\begin{equation}
	IQ(g) = \{ o | o'.G \cap o.G \neq \emptyset \}
\label{eq:ExactMatchQuery}
\end{equation}
Žiūrėti priedą \ref{app:intersectionQuery}.

\subsubsection{Apgaubiančioji užklausa}
Apgaubiančioji užklausa yra tokia užklausa, kuri randa visus objektus $o$, kurių erdvės sritis $o.G$ pilnai apima pateiktąjį objektą $o'$:

\begin{equation}
	EQ(g) = \{ o | (o'.G \cap o.G) = o.G \}
\label{eq:ExactMatchQuery}
\end{equation}
Žiūrėti priedą \ref{app:enclosureQuery}.


\subsubsection{Pilno regiono užklausa}
Pilno regiono užklausa iš esmės yra priešinga apgaubiančiąjai.
Pilno regiono užklausa yra tokia užklausa, kuri randa visus objektus $o$, kurių erdvės sritis $o.G$ pilnai patenka į pateikatjį objektą $o'$:

\begin{equation}
	CQ(g) = \{ o | (o'.G \cap o.G) = g \}
\label{eq:ExactMatchQuery}
\end{equation}
Žiūrėti priedą \ref{app:containmentQuery}.


\subsubsection{Kaimynų užklausa}
Kaimynų užklausa yra tokia užklausa, kuri pagal duotąjį objektą $o'$ suranda visus šio objekto kaimynus, t.y. objektus $o$ kurie turi bendrą kraštinę erdvėje $E^d$:

\begin{equation}
	AQ(g) = \{ o | (o'.G \cap o.G) \neq \emptyset \land o'.G\textdegree \cap o.G\textdegree = \emptyset \}
\label{eq:ExactMatchQuery}
\end{equation}

Čia $o.G\textdegree$ reiškia visus vidinius taškus, kurie priklauso sričiai $o.G$.
Žiūrėti priedą \ref{app:adjacencyQuery}.


\subsubsection{Artimiausio kaimyno užklausa}
Artimiausio kaimyno užklausa yra tokia užklausa, kuri duotam objektui $o'$ randa artimiausią kaimyną (verta pastebėti, kad šiuo atveju kaimynai gali ir neturėti bendrų kraštinių).

\begin{equation}
	NNQ(o') = \{ o | \forall o'' : dist(o'.G, o.G) \leq dist(o'.G, o''.G) \}
\label{eq:ExactMatchQuery}
\end{equation}




\subsection{Daugiamačių duomenų indeksavimo metodų klasifikacija}
Lu ir Ooi \cite{lu1993spatial} \cite{gaede1998multidimensional} \cite{bohm2001searching} apžvelgia įvairius daugiamačių duomenų indeksus, bei pateikia schemą padedančią suprasti jų istoriją (žiūrėti priedą \ref{app:multidimensionalIndexing}).

Autoriai šiuos metodus suskirsto į tris grupes:
\begin{itemize}
	\item Metodai paremti maišos funkcijomis.
	\item Hierarchiniai metodai.
	\item Metodai paremti erdvę užpildančiomis kreivėmis \cite{bader2012space}.
\end{itemize}



\subsubsection{Maišos funkcijomis paremti metodai daugiamačių duomenų indeksavimui}

% generic description
% basically hash tables where buckets are bloks in multidimensional space
% order preserving hashes
% pritaikymo pavyzdžiai

Verta pastebėti, kad dažnai maišos funkcijomis paremti metodai yra tinkami tik griežto atitikmens užklausoms \cite{nievergelt1981grid} \cite{tamminen1982excell}.
Todėl šiame darbe tokie metodai nebus įgyvendinami ir lyginami sistemoje \cite{NeurotechnologyMegamatcherAccelerator}.


\subsubsection{Hierarchiniai metodai daugiamačių duomenų indeksavimui}

Hierarchiniai daugiamačių duomenų indeksai yra paremti dvejetainiais ar aukštesnio atsišakojimo faktoriaus medžiais (trejetainiai, ketvirtainiai...) \cite{gaede1998multidimensional}.
Šie indeksai saugo įrašus ne po vieną, bet grupėmis.
Dažniausiai kiekviena grupė yra randama medžio lapuose (vadinama {\it duomenų viršūne}).
Vidinės viršūnės (vadinamos {\it indekso viršūnėmis}) yra naudojamos kaip kelrodžiai ieškant duomenų viršūnių.
Kiekviena indekso viršūnė nurodo visas duomenų viršūnes, kurios gali būti rastos šios indekso viršūnės pomedyje.

%TODO: image

Užklausų apdorojimas yra skaidomas į du etapus:
\begin{enumerate}
	\item Iteravimas medžiu.
	\item Įrašų filtravimas.
\end{enumerate}
Iteravimo medžiu etape, yra „prabėgamas“ medis nuo šaknies iki lapų ir atrenkamos visos duomenų viršūnės, kurios gali turėti įrašų atitinkančių užklausos kriterijus.
Filtravimo metu atrenkami įrašai esantys iteravimo etape atrinktose grupėse ir tenkinatys užklausos kriterijus \cite{brinkhoff1994multi} \cite{bohm2001searching}.
Verta pastebėti, kad sistemoje \cite{NeurotechnologyMegamatcherAccelerator} užklausų apdorojimo etapai yra labai panašūs.

\paragraph{kd-medis}

Šiame darbe bus aptariamas biometrinių įrašų priskyrimo blokams metodas paremtas kd-medžiu \cite{bentley1979multidimensional}.
Kd-medžių duomenys yra taškai $E^d$ erdvėje.
Kd-medis yra dvejetainis medis, kurio lapuose (duomenų viršūnėse) yra saugomi vienas ar daugiau $d$-mačių taškų.
Kiekviena vidinė medžio viršūnė (indekso viršūnė) dalina erdvę $E^d$ į dvi nepersidengiančias dalis.
Taškai patenkantys į vieną dalį patenka į kairyjį pomedį, o patenkantis į kitą dalį -- į dešinį pomedį.
Erdvės dalinimas į dvi dalis yra parenkamas taip: kiekvienai vidinei viršūnei yra parenkama koordinačių ašis (pvz.: $x$) ir reikšmė (pvz .: $5$).
Tuomet visi taškai, kurių $x$ koordinatės reikšmė yra mažesnė už parinktą reikšmę (pavyzdyje $[-\infty; 5)$), patenka į kairyjį pomedį, likę taškai (pavyzdyje $[5; \infty)$) -- į dešinį pomedį.
Nėra griežtai apibrėžta kokia koordinačių ašis ar reikšmė šioje ašyje turi būti parinkta kiekvienai viršūnei, todėl šiuo tikslu taikomos įvairios euristikos.

%TODO: image

Hierarchiniai metodai daugiamačių duomenų indeksavimui yra plačiai taikomi duomenų bazių \cite{bohm2001searching}, sensorių tinklų \cite{li2003multi}, paveiksliukų paieškoje \cite{silpa2008optimised}, privatumo \cite{hore2012secure} \cite{xiao2010differentially} ir kitose srityse.

\subsubsection{Erdvę užpildančiomis kreivėmis paremti metodai daugiamačių duomenų indeksavimui}
% generic description
% row-wise kreivė, panaši į šiame darbe testuojamą metodą
% kitos kreivės
% pritaikymo pavyzdžiai


\subsection{SP-GIST karkasas}
% sp-gists
% bulk operations on sp-gists
% postgre (min 3)
