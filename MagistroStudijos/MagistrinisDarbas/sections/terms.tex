\sectionnonumnocontent{Terminai}
\begin{enumerate}
\item Biometriniai duomenys - Tai žmogaus kūno savybės leidžiančios jį unikaliai identifikuoti. Pvz.: pirštų atspaudai.
\item Biometrinė identifikacija - Tai kandidato biometrinių duomenų palyginimas su aibe žinomų įrašų siekiant surasti biometrinių duomenų atitikmenis.
\item Biografinis atributas - Tai savybė ar charakteristika siejama su biometrinių duomenų turėtoju (dažniausiai žmogumi) ar jam priklausančiu biometriniu įrašu. Pvz.: žmogaus amžius.
\item Biografinė užklausa - Tai biometrinio identifikavimo užklausa, kurioje identifikacija vykdoma tik su tais biometriniais įrašais, kurių biografiniai atributai atitinka užklausos sąlygas.
\item Biografinė schema - Tai visų biografinių atributų prasmių aibė.
\item Galerija - Tai visų biometrinei identifikavimo sistemai žinomų biometrinių įrašų aibė.
\end{enumerate}


