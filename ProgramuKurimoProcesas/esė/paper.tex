\documentclass[12pt, a4paper, lithuanian, final]{article}

\usepackage{hyperref}
\usepackage{graphicx}
\usepackage{float}
\usepackage{placeins}
\usepackage{gensymb}
\usepackage{amsmath}
\usepackage{textgreek}
\usepackage{mathtools}
\usepackage[obeyFinal]{easy-todo}
\usepackage[utf8]{inputenc}
\def\LTfontencoding{L7x}
\usepackage[\LTfontencoding]{fontenc}
\usepackage[lithuanian]{babel}
%\usepackage{times}

%\renewcommand{\sfdefault}{uhv}
%\renewcommand{\rmdefault}{utm}
%\renewcommand{\ttdefault}{ucr}

\usepackage{VUMIF}


% Titulinio puslapio reikalai
\vumifdept{Programų sistemų katedra}
\vumifpaper{Esė}
\title{Programavimo ir priežiūros veiklų suderinimas mažoje komandoje}
\author{
    4 kurso 1 grupės studentas \\
    Rytis Karpuška
}

\supervisor{lekt. Andrius Adamonis}
\date{Vilnius \\
	2014}


\begin{document}

%titulinis ir turinys
\maketitle

\vumifsectionnonum{Įvadas}

Programinės įrangos kūrimas bei palaikymas dažnai turi labai daug organizacinių klausimų.
Didelėse organizacijose yra sudaromos ir reguliuojamos tvarkos, procesai kaip valdyti ir kontroliuoti visas veiklas reikalingas programinės įrangos kūrimui ir palaikymui atlikti.
Šių procesų sąrašas gali patapti labai ilgas ir todėl siekiant išlaikyti efektyvų darbą, negalime tų pačių procesų taikyti mažos komandos aplinkoje.


Šioje esė pritaikomi programinės įrangos kūrimo principai, pačiam programinės įrangos kūrimo bei palaikymo suderinimo proceso mažoje komandoje, kūrimui.
Tiesa dėl šio darbo laiko ir apimties resursų šie principai taikomi labai supaprastinai.


Iš pradžių apibrėžiami reikalavimai programinės įrangos kūrimui bei palaikymui suderinti mažoje komandoje, paskui "`suprojektuojamas"' galimas priežiūros veiklos procesas, deja nėra galimybės praktiškai įgyvendinti ir išbandyti ("`ištestuoti"') šio proceso.


\paragraph{Reikalavimai mažos komandos procesui}


Šioje esė daroma prielaida, kad komanda jau yra susikūrusi efektyvų programinės įrangos kūrimo procesą, ir jį nagrinėsime juodosios dėžės principu.
Taip pat kuriamas priežiūros procesas, kuris būtų gerai suderinamas su programinės įrangos kūrimo procesu.


Priežiūros procesą būtina sufokusuoti jo tikslui, todėl prie reikalavimų pridedamas programinės įrangos priežiūros tikslas.
Tad fiksuojamas reikalavimas:
\begin{enumerate}
	\item \textit{Priežiūros procesas privalo užtikrinti operatyvų programinės įrangos defektų pašalinimą, svarbius patobulinimus ir kitų pakeitimų atlikimą reikalingą sklandžiam programinės įrangos naudojimui}
	\newcounter{reqList}
	\setcounter{reqList}{\theenumi}
\end{enumerate}


Įprastos komandos programinės įrangos palaikymo periodas yra kur kas ilgesnis, negu komanda gali skirti laiko ir resursų neprisiimdama kitų projektų.
Todėl kuriamas procesas privalo "`palaikyti"' naujos programinės įrangos kūrimo procesą vykstantį lygegriačiai su programinės įrangos priežiūros procesu.
\begin{enumerate}
	\setcounter{enumi}{\thereqList}
	\item \textit{Procesas turi būti efektyviai integruotas su programinės įrangos kūrimo procesu}
	\setcounter{reqList}{\theenumi}
\end{enumerate}


Reikalaujama proceso efektyvumo. Tai yra labai svarbu mažoje komandoje, kurioje nėra didelių žmogiškųjų resursų kiekių.
\begin{enumerate}
	\setcounter{enumi}{\thereqList}
	\item \textit{Procesas negali reikalauti daug resursų proceso organizacinėms veikloms (ang. process overhead)}
	\setcounter{reqList}{\theenumi}
\end{enumerate}


Priežiūros metu gali būti atliekami programinės įrangos pakeitimai, tad procesas privalo užtikrinti, kad šie pakeitimai ištaiso senus defektus ir nesukelia naujų.
\begin{enumerate}
	\setcounter{enumi}{\thereqList}
	\item \textit{Procesas privalo užtikrinti naujų pakeitimų korektiškumą}
	\setcounter{reqList}{\theenumi}
\end{enumerate}


Programinės įrangos priežiūra vyksta atsižvelgiant į dabartinę situaciją su prižiūrimo projekto darbo produktu.
Todėl ir proceso įeities duomenys yra tai kas žinoma tuo metu.
\begin{enumerate}
	\setcounter{enumi}{\thereqList}
	\item \textit{Proceso įeities duomenys yra informacija, surinkta iš produkcinėje aplinkoje veikiančių programų}
	\setcounter{reqList}{\theenumi}
\end{enumerate}


Programinės įrangos vartotojams yra svarbu greitai ir sėkmingai pasiekti savo rezultatų, todėl bet kokie programinės įrangos defektai, ar kiti nesklandumai yra nepatenkintų vartotojų šaltinis.
Šiuo reikalavimu pabrėžiame svarbą kuo greičiau sutvarkyti nesklandumus.
\begin{enumerate}
	\setcounter{enumi}{\thereqList}
	\item \textit{Pageidautina, kad kuriamas priežiūros procesas užtikrintų greitą reakcijos laiką}
	\setcounter{reqList}{\theenumi}
\end{enumerate}


Taip pat laikui bėgant proceso tinkamumas ar pritaikymas gali keistis, todėl svarbu užtikrinti galimybę adaptuoti procesą prie besikeičiančių sąlygų.

\begin{enumerate}
	\setcounter{enumi}{\thereqList}
	\item \textit{Komanda turi peržiūrėti ir įvertinti proceso tinkamumą, ir esant reikalui įgyvendinti pakeitimus procese}
	\setcounter{reqList}{\theenumi}
\end{enumerate}

Patenkinant suformuotus reikalavimus, galima tikėtis efektyvaus ir sklandaus programinės įrangos kūrimo ir priežiūros procesų darbo.

\paragraph{Proceso pasiūlymai}
Didelėse įmonėse, komandose, projektuose, yra populiaru surinkti ne tik defektų pranešimus (ang. bug report), bet ir statistinę informaciją apie tai kaip vartotojai naudojasi programine įranga.
Tačiau net ir mažos komandos, siekdamos atlikti kokybišką programinės įrangos priežiūrą turi pasirūpinti panašios informacijos surinkimu, bei sisteminimu.
Priklausomai nuo komandos resursų, šios informacijos surinkimas, gali apsiriboti paprastais defektų pranešimų surūšiavimais ir sureitingavimais pagal svarbą, o gali būti ir įgyvendintas automatizuotas statistinės vartojimo informacijos surinkimas.
Ir žinoma, kuo detalesnė informacija bus surinkta, tuo tiksliau galima bus atlikti programos priežiūrą.

Šiuo punktu užtikriname reikalavimo 5 patenkinimą

\begin{enumerate}
	\setcounter{enumi}{0}
	\item \textit{Komanda užtikrina duomenų surinkimą apie prižiūrimos programinės įrangos darbą.}
	\setcounter{reqList}{\theenumi}
\end{enumerate}


Formuodami priežūros procesą privalome atsižvelgti į tai, kad komanda tuo pačiu metu yra užsiėmusi kitos programinės įrangos kūrimo darbais, ir šiuos procesus reikia integruoti tarpusavyje taip, kad būtų kuo mažiau resursų (žmogiškųjų, laiko...) švaistymo procesų integracijai.
To reikalaujama reikalavime 2.
Deja priežiūros proceso vykdymui yra būtinas komandos narių laikas, todėl kažkada komanda privalo nusigręžti nuo programinės įrangos kūrimo ir atlikti priežiūros darbus.
Kiekvienas komandos persiorientavimas nuo vieno proceso, prie kito reikalauja nemažai laiko, todėl norėdami minimizuoti laiką praleidžiamą persiorientavimui, turime minimizuoti persiorientavimų skaičių. 
Komandai siūloma susidaryti kriterijus, kada yra būtina pereiti prie priežiūros proceso ir kada grįžti prie programinės įrangos kūrimo proceso.
Šie kriterijai turi būti tokie, kad užtikrintų ne per dažną, bei ne per retą persiorientavimą nuo vieno proceso prie kito.
Verta atkreipti dėmesį, kad reikalavimas 2 (reikalaujantis geros integracijos su kūrimo procesu), bei reikalavimas 6 (reikalaujantis greito proceso atsako) šioje situacijoje yra konfliktiniai.

\begin{enumerate}
	\setcounter{enumi}{\thereqList}
	\item \textit{Komanda sudaro ir laikosi kriterijų apibrėžiančių, kada reikia vykdyti priežiūros, kada kūrimo procesą.}
	\setcounter{reqList}{\theenumi}
\end{enumerate}


Patenkinus persiorientavimo prie priežiūros proceso kriterijus, komanda turėtų išnagrinėti surinktus duomenis apie prižiūrimos programinės įrangos vartojimą ir sudaryti sūrušiuotą problemų sąrašą pagal svarbą.
Įvertinusi laiko sąnaudas kiekvienos iš tų problemų sprendimo įgyvendinimui, komanda turėtų pasirinkti tokį patobulinimą, kurio naudingumo ir laiko sąnaudų santykis yra geriausias.
Reikia atkreipti dėmesį, kad nėra apibrėžiama konkreti procedūra šių problemų svarbumo ar laiko sąnaudų įvertinimui.

\begin{enumerate}
	\setcounter{enumi}{\thereqList}
	\item \textit{Komanda iš surinktų duomenų apie programinės įrangos vartojimą išsirenka pakeitimus, kurie yra naudingiausi pagal įgyvendinimo laiką ir svarbą.}
	\setcounter{reqList}{\theenumi}
\end{enumerate}

Atliekamas pakeitimų įgyvendinimas.
Esė pradžioje buvo padaryta prielaida, kad komanda jau yra susikūrusi efektyvų programinės įrangos kūrimo procesą.
Nors čia nenagrinėjame jo detaliai, bet šiame metode siūloma naudoti tą patį procesą pakeitimų įgyveninimui priežiūros procese.
Tas pats procesas taip pat būtų atsakingas už pakeitimų testavimą (turi patenkinti reikalavimą 4), bei pakeitimų instaliavimą tinkamose aplinkose.

\begin{enumerate}
	\setcounter{enumi}{\thereqList}
	\item \textit{Komanda Įgyvendina pakeitimus.}
	\setcounter{reqList}{\theenumi}
\end{enumerate}

Toliau įvertinami grįžimo prie programinės įrangos kūrimo proceso kriterijai ir esant reikalui užbaigiamas priežiūros ciklas.
Kitu atveju grįžtama prie veiksmo 2.

Taip pat palyginus ilgais laiko intervalais komanda peržiūri proceso efektyvumą, ir įvertina ar pats procesas, bei jo pritaikymas yra tinkami pačiai komandai.
Esant reikalui komanda patobulina pati savo procesą, adaptuoja jį saviems tikslams

\begin{enumerate}
	\setcounter{enumi}{\thereqList}
	\item \textit{Komanda peržiūri patį procesą ir jo vykdymą, esant reikalui įgyvendina pakeitimus.}
	\setcounter{reqList}{\theenumi}
\end{enumerate}


\paragraph{Proceso pritaikymas mažai komandai}

Šis procesas yra apibrėžtas palyginus abstrakčiai ir siekiant efektyvaus jo vykdymo, reikia apsibrėžti keletą savybių laisviems pasirinkimams.
Atsižvelgiant į reikalavimus, vienas iš svarbiausių reikalavimų pritaikant procesą mažai komandai yra reikalavimas 3, teigiantis, kad procesas negali reikalauti daug žmogaus ar kitų resursų pačio proceso organizacinėms veikloms.


Maža komanda turi labai gerai įvertinti kiek ir kokių duomenų reikia surinkinėti iš naudojamos sistemos.
Jeigu komanda neturi automatizuotos duomenų surinkimo sistemos (to pavyzdys galėtų būti bugzilla, jeigu renkami pasiūlymai ar defektų raportai), komanda turėtų apsiriboti tik tiek duomenų, kad jų surinkimas ir saugojimas sudarytų nedidelę laiko dalį bendroje veikloje, arba pasvarstytų apie automatizuotų duomenų surinkimo sistemų įdiegimą.


Dar vienas laisvas ir svarbus pasirinkimas yra problemų išskyrimas iš surinktų duomenų, jų prioritetizavimas ir laiko įvertinimas sprendimo įgyvendinimui.
Mažoje komandoje, neesant labai ilgam problemų sąrašui, galima prioritezavimo klausimus išspręsti formalios ar neformalios žodinės diskusijos metu.


Bendru atveju didžiąją laiko dalį komanda turėtų skirti programavimo procese, arba pakeitimų įgyvendinimo dalyje priežiūros procese, o kitos proceso dalys sudarytų nedidelę išleidžiamo laiko dalį.

Besilaikant šių apribojimų, komanda didžiąją dalį laiko skirs programinės įrangos kūrimui, ir, esant reikalui, pakeitimų įgyvendinimui prižiūrimose programų sistemose.
Šie veiksmai yra veiksmai, kurie suteikia tiesioginius rezultatus, todėl procesą galima vadinti efektyviu.


\paragraph{Išvados}

Šioje esė buvo apmąstyti kokie reikalavimai yra svarbūs mažos komandos priežiūros procesui, pasiūlytas abstraktus proceso variantas, įvertinta galimybė efektyviai vykdyti du procesus lygegriačiai, paliekant lankstumą ir neprisitaikant prie konkrečių komandų ar projektų.
Bei pasiūlytos gairės proceso laisviems kintamiesiems pasirinkti mažoje komandoje.


\end{document}



